\documentclass{article}

\usepackage[final]{neurips}

\usepackage{multicol}
\usepackage{float}
\usepackage[center]{caption}

\usepackage[utf8]{inputenc} % allow utf-8 input
\usepackage[T1]{fontenc}    % use 8-bit T1 fonts
\usepackage{hyperref}       % hyperlinks
\usepackage{url}            % simple URL typesetting
\usepackage{booktabs}       % professional-quality tables
\usepackage{amsfonts}       % blackboard math symbols
\usepackage{nicefrac}       % compact symbols for 1/2, etc.
\usepackage{microtype}      % microtypography
\usepackage{graphicx}
\usepackage{amsmath}
\usepackage{xepersian}

\settextfont{XB Yas.ttf}

\title{
تمرین سوم
}


\author{%
  امیرحسین مهدی‌نژاد\\
  شماره دانشجویی 810800058\\
  \texttt{mahdinejad@ut.ac.ir} \\
}

\begin{document}


\begin{minipage}{0.1\textwidth}% adapt widths of minipages to your needs
\includegraphics[width=1.1cm]{Photos/UT_logo.png}
\end{minipage}%
\hfill%
\begin{minipage}{0.9\textwidth}\raggedleft
دانشکده فنی، دانشگاه تهران\\
الگوریتم‌های تصادفی -  
فروردین
ماه 1400\\
\end{minipage}
% \end{}

\makepertitle

% ----------------------------------------------------------------------
\section*{1}
با توجه به نامساوی مارکوف برای متغیر تصادفی
$X$
و به‌ازای هر
$a$
مثبت و ناصفر داریم:
$$P( X\geq a) \leq \dfrac{E\left[ X\right] }{a}$$
اگر متغیر تصادفی
$X$
را زمان اجرای الگوریتم در نظر بگیریم به‌ازای ثابت
$a$
خواهیم داشت:
$$E\left[ X\right] =O\left( n^{2}\right) \leq an^{2}$$
با در نظر گرفتن هر زمان اجرای
$T=bn^2 (b\geq ka)$
می‌توان نوشت:
$$P( X\geq T) \leq \dfrac{E\left[ X\right] }{T}\leq \dfrac{an^{2}}{T}=\dfrac{an^{2}}{bn^{2}}=\dfrac{a}{b}\leq \frac{1}{k}$$
$$\rightarrow 1-P( X\geq bn^{2}) \geq 1-\dfrac{1}{k}\rightarrow P\left( X\leq bn^{2}\right) \geq 1-\dfrac{1}{k}$$
با در نظر گرفتن مقدار مناسب برای
$k$
این احتمال به ۱ نزدیک خواهد شد و زمان اجرای الگوریتم مذکور در بدترین حالت از مرتبه‌ی 
$n^2$
است.

\rule{\linewidth}{1pt}
% ----------------------------------------------------------------------
\section*{2}
\subsection*{a}
اگر 
$X$
مجموع متغیر‌های تصادفی برنولی یعنی
$X=\sum ^{n}_{i=1}X_{i}$
و همچنین مستقل از
$X_1$
باشد:
$$\begin{aligned}E\left[ XX_{1}\right] =\sum ^{n}_{i=1}\sum ^{n}_{j=1}\left( ij\right) P\left( X=i\right) P\left( Y=j\right) \\ =\left( \sum ^{n}_{i=1}iP\left( X=i\right) \right) \left( \sum ^{n}_{j=1}jP\left( Y=j\right) \right) \end{aligned}$$
$$\rightarrow E\left[ XX_{1}\right] =E\left[ X\right] E\left[ X_{1}\right]$$
همین کار را می‌توان به‌ازای همه‌ی
$X_i$
ها تکرار کرد. در نتیجه:
$$E\left[ X^{2}\right] =\sum ^{n}_{i=1}E\left[ X_{i}X\right] =\sum ^{n}_{i=1}p( X_{i}= 1) E[ X| X_{i}= 1]$$

\subsection*{b}
از نتیجه‌ی بند قبلی استفاده می‌کنیم:
$$E\left[ X^{2}\right] =\sum ^{n}_{i=1}p\left( x_{i}=1\right) E\left[ X| X_{i}=1\right] =\sum ^{n}_{j=0}\begin{pmatrix} n \\ j \end{pmatrix}p^{j}\left( 1-p\right) ^{n-j}j^{2}$$
$$ =\sum ^{n}_{j=0}\dfrac{n!}{\left( n-j\right) !j!}p^{j}\left( 1-p\right) ^{n-j}\left( \left( j^{2}-j\right) +j\right) $$
$$=\sum ^{n}_{j=0}\dfrac{n!}{\left( n-j\right) !j!}p^{j}\left( 1-p\right) ^{n-j}+\sum ^{n}_{j=0}\dfrac{n!}{\left( n-j\right) !j!}p^{j}\left( 1-p\right) ^{n-j}$$
$$ =n\left( n-1\right) p^{2}\sum ^{n}_{j=2}\dfrac{\left( n-2\right) !}{\left( n-j\right) !\left( j-2\right) !}p^{j-2}\left( 1-p\right) ^{n-j}+np\sum ^{n}_{j=1}\dfrac{\left( n-1\right) !}{\left( n-j\right) !\left( j-1\right) !}p^{j-1}\left( 1-p\right) ^{n-j}$$
$$ =n\left( n-1\right) p^{2}+np$$
همچنین واریانس متغیر تصادفی دوجمله‌ای
$X$
با پارامترهای
$n, p$
به این صورت بدست می‌آید:
$$Var\left[ X\right] =E\left[ X^{2}\right] -E\left[ X\right] ^{2}=n\left( n-1\right) p^{2}+np-n^{2}p^{2}=np\left( 1-p\right)$$

\rule{\linewidth}{1pt}
% ----------------------------------------------------------------------
\section*{3}
در نظر بگیریم
$X_{ij}=I\left( a_{i},aj\right)$
و همچنین
$S=\sum _{i <j}X_{ij}$
در این صورت با توجه به اینکه هر جابجایی و از بین رفتن وارونگی،
$S$
را یک واحد کاهش می‌دهد، کافیست مقدار
\lr{expected}
را بدست آوریم:
$$E\left[ S\right] =E\left[ \sum _{i <j}X_{ij}\right] =\sum _{i <j}E\left[ X_{ij}\right] =\sum _{i <j}p\left( a_{i} >a_{j}\right) \\ =\dfrac{1}{2}\sum ^{n}_{i=1}\left( n-i\right) =\dfrac{1}{2}\sum ^{n-1}_{i=0}i=\dfrac{\left( n-1\right) n}{4}$$
به همین ترتیب با در نظر گرفتن احتمالات
$p( a_{i}^{2} > a_{j}^{2})$
برای 
$S^2$
خواهیم داشت:
$$E\left[ S^{2}\right] =E\left[ \sum _{i <j}X_{ij}^{2}\right] =\sum _{i <j}X_{ij}^{2}=\dfrac{1}{2}\sum ^{n}_{i=1}\left( n-i\right) =\dfrac{n\left( n-1\right) }{4}$$
حال واریانس را محاسبه می‌کنیم:
$$Var\left[ S\right] =E\left[ S^{2}\right] -E\left[ S\right] ^{2}\\ =\dfrac{n\left( n-1\right) }{4}-\left( \dfrac{n\left( n-1\right) }{4}\right) ^{2}\\ =\dfrac{n\left( n-1\right) }{4}\left( 1-\dfrac{n\left( n-1\right) }{4}\right)$$

\rule{\linewidth}{1pt}
% ----------------------------------------------------------------------

\section*{4}
برای توزیع اشیاء با
$k+1$
درجه‌ی کیفیت خواهیم داشت:
$$\left| x^{r}\left( \dfrac{k+1}{k+1+x^{2}}\right) ^{1+\dfrac{k}{2}}\right| \geq \dfrac{\left| x\right| ^{r}}{\left( 1+x^{2}\right) ^{1+\dfrac{k}{2}}}$$
همگرایی انتگرال
\lr{r}
مین
\lr{moment}
برای
$1 \leq r \leq k$
به این صورت خواهد بود:
$$\begin{aligned}\int ^{\infty }_{-\infty }\left| x^{r}N\left( \dfrac{k+1}{k+1+x^{2}}\right) ^{1+\dfrac{k}{2}}\right| dx\geq N\int _{-\infty }^{\infty }\dfrac{\left| x\right| ^{r}}{\left( 1+x^{2}\right) ^{1+\dfrac{k}{2}}}dx\\ =2N\int _{0}^{\infty }\dfrac{x^{r}}{\left( 1+x^{2}\right) ^{1+\dfrac{k}{2}}}dx\end{aligned}$$
که اگر
$r \leq k$
باشد، فقط در
$\infty$
همگرا می‌شود.

\rule{\linewidth}{1pt}
% ----------------------------------------------------------------------
\section*{5}
\subsection*{a}
دو بیت انتخاب می‌کنیم. اگر
$V_1$
مقدار بیت اول و
$V_2$
مقدار بیت دوم باشد، این دو بیت از یکدیگر مستقل بوده و احتمال صفر یا یک بودن هرکدام
$0.5$
است. پس:
$$\begin{aligned}P\left( Y_{i}=0\right) =P\left( V_{1}=V_{2}\right) =P\left( (V_{1}=0\& V_{2}=0) \text{یا} (V_{1}=1\& V_{2}=1)\right) \\ =0.5^{2}+\left( 1-0.5\right) ^{2}=0.5\end{aligned}$$

\subsection*{b}
اگر زوج
$j$
بیت‌های اول یکسان با زوج
$i$
داشته باشند یعنی مستقل نیستند.
فرض کنیم
$V_{1,i}$
مقدار اولین بیت در زوج
$i$
و به همین ترتیب
$V_{2,i}$
مقدار دومین بیت در همان زوج باشد. با در نظر گرفتن
$V_{1,i} = V_{1,j}$
اولین بیت در هر دو زوج یکسان است و داریم:
$$P( V_{1,i}= V_{2,i}) =0.5^{2}+\left( 1-0.5\right) ^{2}=0.5$$
$$P( V_{1,j}= V_{2,j}| V_{1,i}= V_{2,i}) =P( V_{1,i}=V_{2,i}= V_{2,j}) =0.5^{3}+\left( 1-0.5\right) ^{3}=0.25$$
در حالی که:
$$P( Y_{i}=0\& Y_{j}= 0) =P( Y_{j}= 0| Y_{i}= 0) P( Y_{i}= 0) =0.25\times 0.5=0.125$$
این دو مقدار برابر نشدند.

\subsection*{c}
با توجه به:
$$E\left[ Y_{i}Y_{j}\right] =1\times P\left( Y_{i}=1\& Y_{j}=1\right)$$
$$E\left[ Y_{i}\right] =1\times P\left( Y_{i}=1\right)$$
و با در نظر گرفتن برابری اولین بیت‌ها:
$$\begin{aligned}P\left( Y_{i}=1\& Y_{j}=1\right) =P\left( V_{1,j}\neq V_{2,j}\& V_{1,i}\neq V_{2,i}\right) \\ =P\left( V_{2,j}\neq V_{1,i}\& V_{2,i}\neq V_{1,i}\right) \end{aligned}$$
از طرفی
$V_{2,j}$
و
$V_{2,i}$
مستقل‌اند؛ لذا:
$$\begin{aligned}P\left( V_{2,j}\neq V_{1,i}\& V_{2,i}\neq V_{1,i}\right) =P\left( V_{2,j}\neq V_{1,i}\right) P\left( V_{2,i}\neq V_{1,i}\right) \\ =P\left( V_{2,j}\neq V_{1,i}\right) P\left( Y_{i}=1\right) \end{aligned}$$
به این ترتیب از مستقل بودن
$V_{2,j}$
و
$V_{1,i}$
داریم:
$$P( V_{2,j}\neq V_{1,i}) = P(Y_j = 1) 0.5\left( 1-0.5\right) +\left( 1-0\cdot 5\right) 0.5=0.5$$
در نهایت خواسته‌ی سوال نتیجه می‌شود:
$$p( Y_{i}=1\& Y_{i}= 1) =P( Y_{i}= 1) P( Y_{j}= 1)$$

\subsection*{d}
با توجه به استقلال متغیرها می‌توان نوشت:
$$\begin{aligned}Var\left[ Y\right] =\sum ^{m}_{i=1}Var\left( Y_{i}\right) =\sum ^{m}_{i=1}E\left( Y_{i}\right) ^{2}-E\left( Y_{i}\right) ^{2}\\ =\sum ^{m}_{i=1}\left( 0^{2}\dfrac{1}{2}+1^{2}\dfrac{1}{2}\right) -\left( 0\dfrac{1}{2}+1\dfrac{1}{2}\right) ^{2}=\sum ^{m}_{i=1}\dfrac{1}{4}\\ =\dfrac{m}{4}\end{aligned}$$

\subsection*{e}
با توجه به چبی‌شف داریم:
$$P[ \left| Y-E\left( Y\right) \right| \geq n] \leq \dfrac{Var\left( Y\right) }{n^{2}}=\dfrac{1}{8}\times \dfrac{n\left( n-1\right) }{n^{2}}\leq \dfrac{1}{8}$$

\rule{\linewidth}{1pt}
% ----------------------------------------------------------------------
\section*{6}
\subsection*{a}
اگر
$\sigma \left[ X\right] =0$
باشد می‌توان گفت
$X = E \left[ X\right]$
و به‌ازای هر مقدار
$t$
داریم
$P(X - E \left[ X\right] \geq t \sigma \left[ X\right]) = 1$.

به‌ازای 
$\sigma \left[ X\right] > 0$
متغیر تصادفی نرمال‌سازی‌شده‌ی 
$Y$
را به این صورت تعریف می‌کنیم:
$Y = \frac{X - E \left[ X\right]}{\sigma (X)}$.
لذا خواهیم داشت:
$$E\left[ Y\right] =\dfrac{E\left[ X\right] -E\left[ X\right] }{\sigma \left( X\right) }=0$$
$$E\left[ Y^{2}\right] =\dfrac{E\left[ \left( X-E\left[ X\right] \right) ^{2}\right] }{\sigma ^{2}\left( X\right) }=\dfrac{\sigma ^{2}\left( X\right) }{\sigma ^{2}\left( X\right) }=1$$
پس با در نظر داشتن
$t > 0$
می‌توان گفت:
$$X-E\left[ X\right] \geq t\sigma \left( X\right) \Leftrightarrow Y\geq t\Leftrightarrow tY\geq t^{2}\Leftrightarrow 1+tY\geq 1+t^{2}$$
در ادامه با توجه به نامساوی مارکوف می‌توان به خواسته‌ی سوال رسید:
$$\begin{aligned}P\left( X-E\left[ X\right] \geq t\sigma \left( X\right) \right) =p\left( 1+tY\geq 1+t^{2}\right) \leq P\left( \left| 1+tY\right| \geq 1+t^{2}\right) \\ \leq \dfrac{E\left[ \left( 1+tY\right) ^{2}\right] }{\left( 1+t^{2}\right) ^{2}}=\dfrac{E\left[ 1+rtY+t^{2}Y^{2}\right] }{\left( 1+t^{2}\right) ^{2}}=\dfrac{1+2tE\left[ Y\right] +t^{2}E\left[ Y^{2}\right] }{\left( 1+t^{2}\right) ^{2}}\\ =\dfrac{1+t^{2}}{\left( 1+t^{2}\right) ^{2}}=\dfrac{1}{1+t^{2}}\end{aligned}$$

\subsection*{b}
از نتیجه‌ی بند قبلی استفاده می‌کنیم و برای دو متغیر تصادفی
$X$
و
$-X$
خواهیم داشت:
$$P\left( \left| X-E\left[ X\right] \right| \geq t\sigma \left[ X\right] \right) =P\left( X-E\left[ X\right] \geq t\sigma \left[ X\right] \cup -X+E\left[ X\right] \geq t\sigma \left[ X\right] \right) $$
$$=P( X-E\left[ X\right] \geq t\sigma \left[ X\right] ) +P( -X+E\left[ X\right] \geq t\sigma \left[ X\right] )$$
$$=P( X-E\left[ X\right] \geq t\sigma \left[ X\right] ) +P( -X-E\left[ -X\right] \geq t\sigma \left[ X\right] )$$
$$\leq \dfrac{1}{1+t^{2}}+\dfrac{1}{1+t^{2}}=\dfrac{2}{1+t^{2}}$$
که همان خواسته‌ی سوال است.

\rule{\linewidth}{1pt}
% ----------------------------------------------------------------------
\section*{7}
با توجه به مستقل بودن
$T_i$
ها واریانس را محاسبه می‌کنیم:
$$\begin{aligned}Var\left[ T_{n}\right] =\sum ^{n-1}_{i=0}Var\left[ T_{i}\right] =\sum ^{n-1}_{i=0}\dfrac{1-p_{i}}{p_{i}^{2}}=\sum ^{n-1}_{i=0}\dfrac{ni}{\left( n-i\right) ^{2}}\\ =\sum ^{n}_{i=1}\dfrac{n\left( n-i\right) }{i^{2}}\leq n^{2}\sum ^{n}_{i=1}\dfrac{1}{i^{2}}\leq \dfrac{\pi ^{2}n^{2}}{6}\end{aligned}$$
از نامساوی چبی‌شف داریم:
$$\begin{aligned}P\left[ \left| T_{n}-E\left[ T_{n}\right] \right| \geq cE\left[ T_{n}\right] \right] =P\left[ \left| T_{n}-E\left[ T_{n}\right] \right| \geq cnH_{n}\right] \\ \leq \dfrac{Var\left[ T_{n}\right] }{\left( cnH_{n}\right) ^{2}}\leq \dfrac{\pi ^{2}}{6c^{2}H_{n}^{2}}\end{aligned}$$
اگر 
$B_b$
تعداد توپ‌های داخل سطل
$b$
بوده و 
$m=n\log n+cn$
فرض شود طبق کران چرنوف:
$$P\left[ B_{b}\leq 0\right] =P\left[ B_{b}\leq \left( 1-1\right) E\left[ B_{b}\right] \right] \leq \exp \left( \dfrac{-E\left[ Bb\right] }{2}\right) \leq \exp \left( \dfrac{-\left( \log n+c\right) }{2}\right)$$
که با محاسبه‌ی مستقیم می‌توان کران بهتری برای آن یافت:
$$P\left( \text{سطلی خالی باشد}\right) \leq nP\left[ B_{b}\leq 0\right] =\exp \left( \log n-\dfrac{\left( \log n+c\right) }{2}\right)$$
$$P\left[ B_{b}\leq 0\right] =\left( 1-\dfrac{1}{n}\right) ^{m}\leq e^{-\dfrac{m}{n}}=e^{-\dfrac{c}{n}}
$$
لذا کران چرنوف برای وقتی که احتمالات کوچک هستند و دنبال کران
\lr{tight}
نیستیم مناسب‌تر است.

\rule{\linewidth}{1pt}
% ----------------------------------------------------------------------
\section*{8}
طبق نامساوی چبی‌شف برای هر
$a > 0$
داریم:
$$P( \left| X-E\left[ X\right] \right| \geq a) \leq \dfrac{Var\left[ X\right] }{a^{2}}
$$
از طرفی با جایگذاری مقادیر صورت سوال داریم:
$$\begin{aligned}P\left( \left| X-E\left[ X\right] \right| \geq \beta \right) =P\left( \left| X-E\left[ X\right] \right| ^{k}\geq \beta ^{k}\right) \\ \leq \dfrac{E\left[ \left| x-E\left[ x\right] \right| ^{k}\right] }{\beta ^{k}}\end{aligned}$$
که تفاوت آن با خواسته‌ی سوال در علامت قدرمطلق سمت راست نامساوی است که با توجه به فرض سوال (زوج بودن
$k$
) می‌توان آن را به این صورت نیز نوشت.

\rule{\linewidth}{1pt}
% ----------------------------------------------------------------------
\section*{10}
با مرتبه‌ی
$n$
مجموعه‌های
$S_1$
و
$S_2$
ساخته می‌شود. اگر عددی که داریم بزرگتر یا مساوی
$k$
بود قطعا متعلق به
$S_1$
و اگر از
$k-1$
کمتر بود به
$S_2$
خواهد بود و در صورتی که خود
$k$
باشد همان عدد تصادفی است که انتخاب کردیم. پس در هر مرحله کاری از مرتبه‌ی تعداد اعضا انجام شده و تعداد کم می‌شود.

امید ریاضی اعدادی که کم می‌شود (مجموعه‌ی باقی‌مانده) نصف اعدادی است که تا این مرحله در اختیار داریم یعنی رابطه‌ی بازگشتی آن مشابه مرتب‌سازی خواهد شد:
$$E \left[ T\left( n\right) \right] =E \left[ T\left( \dfrac{n}{2}\right) \right] +O\left( n\right)$$
که از 
$2n$
کوچکتر است.

\rule{\linewidth}{1pt}
% ----------------------------------------------------------------------
به دلیل زیاد بودن عبارت‌های ریاضی ممکن است در بعضی از قسمت‌های محاسبات
$p$
به جای
$P$
(و خطاهایی از این دست)
نوشته شده و اصلاح نشده باشد.
\end{document}