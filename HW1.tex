\documentclass{article}

\usepackage[final]{neurips}

\usepackage{multicol}
\usepackage{float}
\usepackage[center]{caption}

\usepackage[utf8]{inputenc} % allow utf-8 input
\usepackage[T1]{fontenc}    % use 8-bit T1 fonts
\usepackage{hyperref}       % hyperlinks
\usepackage{url}            % simple URL typesetting
\usepackage{booktabs}       % professional-quality tables
\usepackage{amsfonts}       % blackboard math symbols
\usepackage{nicefrac}       % compact symbols for 1/2, etc.
\usepackage{microtype}      % microtypography
\usepackage{graphicx}
\usepackage{amsmath}
\usepackage{xepersian}

\settextfont{XB Yas.ttf}

\title{
تمرین اول
}


\author{%
  امیرحسین مهدی‌نژاد\\
  شماره دانشجویی 810800058\\
  \texttt{mahdinejad@ut.ac.ir} \\
}

\begin{document}


\begin{minipage}{0.1\textwidth}% adapt widths of minipages to your needs
\includegraphics[width=1.1cm]{Photos/UT_logo.png}
\end{minipage}%
\hfill%
\begin{minipage}{0.9\textwidth}\raggedleft
دانشکده فنی، دانشگاه تهران\\
الگوریتم‌های تصادفی -  
اسفند
ماه 1400\\
\end{minipage}
% \end{}

\makepertitle

% ----------------------------------------------------------------------
\section*{1}
\subsection*{a}
از کل حالات
(۶ به توان ۳ حالت)
فقط در ۶ حالت، به ازای هر عدد ممکن روی تاس، برابری اتفاق خواهد افتاد. لذا:
$$ P(\text{problem}_{1.a}) = \frac{6}{6^3} = \frac{1}{6^2} $$

\subsection*{b}
برای حالات مطلوب، انتخاب می‌کنیم که کدام دو تاس ابتدا مشابه شوند، که عددی که رو می‌آید ۶ حالت دارد و تاس سوم نیز به ۵ حالت می‌تواند با آن‌ها تفاوت داشته باشد. لذا:
$$ P(\text{problem}_{1.b}) = \frac{\binom{3}{2} \times 5 \times 6}{6^3} = \frac{5}{12} $$

\subsection*{c}
در اینجا احتمال شرطی بردن در حالی که ۲ تاس در پرتاب اول یکسان آمده باشند خواسته شده است. یعنی کافیست در پرتاب دوم، تاس سوم با آن دو تاس یکسان شود:
$$ P(\text{win}|\text{\lr{same \# on two dice}}) = \frac{1}{6} $$

\subsection*{d}
با توجه به حالت‌های ذکر شده در صورت سوال، به سه مدل ممکن است این اتفاق افتاده باشد:
\begin{itemize}
    \item 
    در پرتاب اول هر سه تاس یکسان باشند
    \item
    در پرتاب اول دو تاس یکسان شده باشند که همان نتیجه‌ی بند
    \lr{b}
    بدست می‌آید
    \item
    در پرتاب اول هر سه تاس متفاوت باشند و دوباره پرتاب انجام شود
\end{itemize}
به ترتیب از چپ به راست حالات را نوشته و جمع کردیم:
$$ P(\text{win}) = \frac{1}{6^2} + \frac{5}{12} \times \frac{1}{6} + \frac{6 \times 5 \times 4}{6^3} \times \frac{1}{6^2}$$
\rule{\linewidth}{1pt}
% ----------------------------------------------------------------------

\section*{2}
\subsection*{a}
تعداد یکسان‌سازی‌ها
$2(n-2)$
است. مشابه توضیحات صفحه‌ی ۱۷ کتاب و با توجه به اینکه الگوریتم دو بار اجرا می‌شود، داریم:
$$ P(\text{problem}_{2.a}) = \left(1 - \frac{2}{n(n-1)}\right)^2 $$

\subsection*{b}
تعداد یکسان‌سازی‌ها این بار
$(n-k) + l(k-2)$
است. مشابه توضیحات صفحه‌ی ۱۷ کتاب، داریم:
$$ P(\text{problem}_{2.b}) = \frac{k}{n} \times \frac{k-1}{n-1} \times \left(1 - \left(1 - \frac{2}{k(k-1)} \right)^l\right) \leq \frac{k^2}{n^2} \left(1 - e^{-\frac{2l}{k^2}}\right) $$

\subsection*{c}
مقدار
$l$
را با توجه به تعداد یکسان‌سازی‌های بند
\lr{b}
و برابر قرار دادن آن با تعداد یکسان‌سازی‌های بند
\lr{a}
بدست می‌آوریم:
$$ l = \frac{n+k-4}{k-2} $$
که با جایگذاری در نتیجه‌ی بند قبل به صورت مقابل خواهد شد:
$$\frac{k^2}{n^2}\left(1 - e^{-\frac{n+k}{k^3}}\right)$$
که با مقدار
$k = n^{1/3}$
بیشینه شده و احتمال موفقیت حداقل از مرتبه‌ی
$\frac{n^{\frac{2}{3}}}{n^2} = n^{\frac{-4}{3}}$
است.

\rule{\linewidth}{1pt}
% ----------------------------------------------------------------------

\section*{3}
\subsection*{a}
برای مینی‌مم، فرایندی تصادفی تعریف می‌کنیم که اگر اولی شیر بود دومی قطعا خط و سومی قطعا شیر باشد، اگر اولی خط بود دومی قطعا شیر و سومی قطعا خط باشد؛ یعنی سکه‌ها بدین شکل وابسته اند. 
در اینجا احتمال شیر آمدن سکه‌ی دوم
$\frac{1}{2}$
و احتمال اینکه هر سه سکه شیر بشوند صفر است.

برای ماکزیمم نمی‌توان احتمال
\lr{HHH}
را یک در نظر گرفت چون در این صورت برای هر سکه‌ای احتمال شیر آمدن با خط آمدن برابر نیست. پس کران بالای مناسب در پرتاب سکه‌ها بدین شکل بدست می‌آید:

احتمال شیر آمدن سکه‌ی اول
$\frac{1}{2}$
است و از آن بیشتر نمی‌شود. پس حالت‌هایی که سکه‌ی اول شیر یا خط امده باشد جمع احتمال‌های یکسانی دارند پس بهترین کران برای
\lr{HHH}
برابر با 
$\frac{1}{2}$
است.

\subsection*{b}
اینجا بر خلاف سوال قبلی که فقط وابستگی سکه‌های دوم و سوم به سکه‌ی اول را در نظر گرفتیم، پرتاب‌ها دو به دو وابسته تعریف می‌شوند. مثلا اگر دو سکه‌ی اول مثل هم بودند سکه‌ی سوم خط و در غیر این صورت شیر بیاید. پس مینی‌مم برای
\lr{HHH}
باز هم صفر است.

برای ماکزیمم اگر دو سکه‌ی اول مستقل باشند و با احتمال
$\frac{1}{4}$
شیر بیایند، باز هم فرآیند به این شکل تعریف شود که در صورت برابری دو سکه‌ی اول، سومی شیر و در غیر این صورت خط باشد و کران بالا همان
$\frac{1}{4}$
است.

\end{document}